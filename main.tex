%TC:envir hscode [] ignore
\documentclass[a4paper, twocolumn, 10pt]{extarticle}

% Tiny borders should be default
\usepackage[a4paper, total={7in, 10.25in}]{geometry}

\setlength {\marginparwidth }{2cm}
\usepackage{todonotes}
% Font shit
%\usepackage{fontspec}
%\usepackage{xunicode}
%\usepackage{xcolor}

\usepackage{libertine}
\usepackage{url}
\newmuskip\codemuskip
\codemuskip=4.0mu plus 2.0mu minus 2.0mu\relax
\newcommand\codeskip{\mskip\codemuskip}%
\let\codefont\textsf
\newcommand\sub[1]{\ensuremath{_{\text{#1}}}}

\usepackage{xspace}
\newcommand\keyw[1]{{\codefont{\textbf{#1}}}}
\newcommand\id[1]{\Varid{#1}}
\newcommand\idsym[1]{\mathbin{\id{#1}}}
\newcommand{\vertrule}[1][1.0ex]{\rule[-0.0ex]{.45pt}{#1}}

\usepackage[sort, numbers]{natbib}
\bibliographystyle{ACM-Reference-Format}
\defcitealias{embedding}{\textit{Folding Domain-Specific Languages: Deep and Shallow Embeddings}}

\usepackage{enumitem}
\usepackage{hyphenat}
\usepackage[switch]{lineno}
%\linenumbers


\title{\vspace{-10mm}An Overview of \citetalias{embedding}\vspace{-4mm}}
\author{Riley Evans (re17105)}
\date{\vspace{-3mm}}


%% ODER: format ==         = "\mathrel{==}"
%% ODER: format /=         = "\neq "
%
%
\makeatletter
\@ifundefined{lhs2tex.lhs2tex.sty.read}%
  {\@namedef{lhs2tex.lhs2tex.sty.read}{}%
   \newcommand\SkipToFmtEnd{}%
   \newcommand\EndFmtInput{}%
   \long\def\SkipToFmtEnd#1\EndFmtInput{}%
  }\SkipToFmtEnd

\newcommand\ReadOnlyOnce[1]{\@ifundefined{#1}{\@namedef{#1}{}}\SkipToFmtEnd}
\usepackage{amstext}
\usepackage{amssymb}
\usepackage{stmaryrd}
\DeclareFontFamily{OT1}{cmtex}{}
\DeclareFontShape{OT1}{cmtex}{m}{n}
  {<5><6><7><8>cmtex8
   <9>cmtex9
   <10><10.95><12><14.4><17.28><20.74><24.88>cmtex10}{}
\DeclareFontShape{OT1}{cmtex}{m}{it}
  {<-> ssub * cmtt/m/it}{}
\newcommand{\texfamily}{\fontfamily{cmtex}\selectfont}
\DeclareFontShape{OT1}{cmtt}{bx}{n}
  {<5><6><7><8>cmtt8
   <9>cmbtt9
   <10><10.95><12><14.4><17.28><20.74><24.88>cmbtt10}{}
\DeclareFontShape{OT1}{cmtex}{bx}{n}
  {<-> ssub * cmtt/bx/n}{}
\newcommand{\tex}[1]{\text{\texfamily#1}}	% NEU

\newcommand{\Sp}{\hskip.33334em\relax}


\newcommand{\Conid}[1]{\mathit{#1}}
\newcommand{\Varid}[1]{\mathit{#1}}
\newcommand{\anonymous}{\kern0.06em \vbox{\hrule\@width.5em}}
\newcommand{\plus}{\mathbin{+\!\!\!+}}
\newcommand{\bind}{\mathbin{>\!\!\!>\mkern-6.7mu=}}
\newcommand{\rbind}{\mathbin{=\mkern-6.7mu<\!\!\!<}}% suggested by Neil Mitchell
\newcommand{\sequ}{\mathbin{>\!\!\!>}}
\renewcommand{\leq}{\leqslant}
\renewcommand{\geq}{\geqslant}
\usepackage{polytable}

%mathindent has to be defined
\@ifundefined{mathindent}%
  {\newdimen\mathindent\mathindent\leftmargini}%
  {}%

\def\resethooks{%
  \global\let\SaveRestoreHook\empty
  \global\let\ColumnHook\empty}
\newcommand*{\savecolumns}[1][default]%
  {\g@addto@macro\SaveRestoreHook{\savecolumns[#1]}}
\newcommand*{\restorecolumns}[1][default]%
  {\g@addto@macro\SaveRestoreHook{\restorecolumns[#1]}}
\newcommand*{\aligncolumn}[2]%
  {\g@addto@macro\ColumnHook{\column{#1}{#2}}}

\resethooks

\newcommand{\onelinecommentchars}{\quad-{}- }
\newcommand{\commentbeginchars}{\enskip\{-}
\newcommand{\commentendchars}{-\}\enskip}

\newcommand{\visiblecomments}{%
  \let\onelinecomment=\onelinecommentchars
  \let\commentbegin=\commentbeginchars
  \let\commentend=\commentendchars}

\newcommand{\invisiblecomments}{%
  \let\onelinecomment=\empty
  \let\commentbegin=\empty
  \let\commentend=\empty}

\visiblecomments

\newlength{\blanklineskip}
\setlength{\blanklineskip}{0.66084ex}

\newcommand{\hsindent}[1]{\quad}% default is fixed indentation
\let\hspre\empty
\let\hspost\empty
\newcommand{\NB}{\textbf{NB}}
\newcommand{\Todo}[1]{$\langle$\textbf{To do:}~#1$\rangle$}

\EndFmtInput
\makeatother
%
%
%
%
%
%
% This package provides two environments suitable to take the place
% of hscode, called "plainhscode" and "arrayhscode". 
%
% The plain environment surrounds each code block by vertical space,
% and it uses \abovedisplayskip and \belowdisplayskip to get spacing
% similar to formulas. Note that if these dimensions are changed,
% the spacing around displayed math formulas changes as well.
% All code is indented using \leftskip.
%
% Changed 19.08.2004 to reflect changes in colorcode. Should work with
% CodeGroup.sty.
%
\ReadOnlyOnce{polycode.fmt}%
\makeatletter

\newcommand{\hsnewpar}[1]%
  {{\parskip=0pt\parindent=0pt\par\vskip #1\noindent}}

% can be used, for instance, to redefine the code size, by setting the
% command to \small or something alike
\newcommand{\hscodestyle}{}

% The command \sethscode can be used to switch the code formatting
% behaviour by mapping the hscode environment in the subst directive
% to a new LaTeX environment.

\newcommand{\sethscode}[1]%
  {\expandafter\let\expandafter\hscode\csname #1\endcsname
   \expandafter\let\expandafter\endhscode\csname end#1\endcsname}

% "compatibility" mode restores the non-polycode.fmt layout.

\newenvironment{compathscode}%
  {\par\noindent
   \advance\leftskip\mathindent
   \hscodestyle
   \let\\=\@normalcr
   \let\hspre\(\let\hspost\)%
   \pboxed}%
  {\endpboxed\)%
   \par\noindent
   \ignorespacesafterend}

\newcommand{\compaths}{\sethscode{compathscode}}

% "plain" mode is the proposed default.
% It should now work with \centering.
% This required some changes. The old version
% is still available for reference as oldplainhscode.

\newenvironment{plainhscode}%
  {\hsnewpar\abovedisplayskip
   \advance\leftskip\mathindent
   \hscodestyle
   \let\hspre\(\let\hspost\)%
   \pboxed}%
  {\endpboxed%
   \hsnewpar\belowdisplayskip
   \ignorespacesafterend}

\newenvironment{oldplainhscode}%
  {\hsnewpar\abovedisplayskip
   \advance\leftskip\mathindent
   \hscodestyle
   \let\\=\@normalcr
   \(\pboxed}%
  {\endpboxed\)%
   \hsnewpar\belowdisplayskip
   \ignorespacesafterend}

% Here, we make plainhscode the default environment.

\newcommand{\plainhs}{\sethscode{plainhscode}}
\newcommand{\oldplainhs}{\sethscode{oldplainhscode}}
\plainhs

% The arrayhscode is like plain, but makes use of polytable's
% parray environment which disallows page breaks in code blocks.

\newenvironment{arrayhscode}%
  {\hsnewpar\abovedisplayskip
   \advance\leftskip\mathindent
   \hscodestyle
   \let\\=\@normalcr
   \(\parray}%
  {\endparray\)%
   \hsnewpar\belowdisplayskip
   \ignorespacesafterend}

\newcommand{\arrayhs}{\sethscode{arrayhscode}}

% The mathhscode environment also makes use of polytable's parray 
% environment. It is supposed to be used only inside math mode 
% (I used it to typeset the type rules in my thesis).

\newenvironment{mathhscode}%
  {\parray}{\endparray}

\newcommand{\mathhs}{\sethscode{mathhscode}}

% texths is similar to mathhs, but works in text mode.

\newenvironment{texthscode}%
  {\(\parray}{\endparray\)}

\newcommand{\texths}{\sethscode{texthscode}}

% The framed environment places code in a framed box.

\def\codeframewidth{\arrayrulewidth}
\RequirePackage{calc}

\newenvironment{framedhscode}%
  {\parskip=\abovedisplayskip\par\noindent
   \hscodestyle
   \arrayrulewidth=\codeframewidth
   \tabular{@{}|p{\linewidth-2\arraycolsep-2\arrayrulewidth-2pt}|@{}}%
   \hline\framedhslinecorrect\\{-1.5ex}%
   \let\endoflinesave=\\
   \let\\=\@normalcr
   \(\pboxed}%
  {\endpboxed\)%
   \framedhslinecorrect\endoflinesave{.5ex}\hline
   \endtabular
   \parskip=\belowdisplayskip\par\noindent
   \ignorespacesafterend}

\newcommand{\framedhslinecorrect}[2]%
  {#1[#2]}

\newcommand{\framedhs}{\sethscode{framedhscode}}

% The inlinehscode environment is an experimental environment
% that can be used to typeset displayed code inline.

\newenvironment{inlinehscode}%
  {\(\def\column##1##2{}%
   \let\>\undefined\let\<\undefined\let\\\undefined
   \newcommand\>[1][]{}\newcommand\<[1][]{}\newcommand\\[1][]{}%
   \def\fromto##1##2##3{##3}%
   \def\nextline{}}{\) }%

\newcommand{\inlinehs}{\sethscode{inlinehscode}}

% The joincode environment is a separate environment that
% can be used to surround and thereby connect multiple code
% blocks.

\newenvironment{joincode}%
  {\let\orighscode=\hscode
   \let\origendhscode=\endhscode
   \def\endhscode{\def\hscode{\endgroup\def\@currenvir{hscode}\\}\begingroup}
   %\let\SaveRestoreHook=\empty
   %\let\ColumnHook=\empty
   %\let\resethooks=\empty
   \orighscode\def\hscode{\endgroup\def\@currenvir{hscode}}}%
  {\origendhscode
   \global\let\hscode=\orighscode
   \global\let\endhscode=\origendhscode}%

\makeatother
\EndFmtInput
%
%
%
% First, let's redefine the forall, and the dot.
%
%
% This is made in such a way that after a forall, the next
% dot will be printed as a period, otherwise the formatting
% of `comp_` is used. By redefining `comp_`, as suitable
% composition operator can be chosen. Similarly, period_
% is used for the period.
%
\ReadOnlyOnce{forall.fmt}%
\makeatletter

% The HaskellResetHook is a list to which things can
% be added that reset the Haskell state to the beginning.
% This is to recover from states where the hacked intelligence
% is not sufficient.

\let\HaskellResetHook\empty
\newcommand*{\AtHaskellReset}[1]{%
  \g@addto@macro\HaskellResetHook{#1}}
\newcommand*{\HaskellReset}{\HaskellResetHook}

\global\let\hsforallread\empty

\newcommand\hsforall{\global\let\hsdot=\hsperiodonce}
\newcommand*\hsperiodonce[2]{#2\global\let\hsdot=\hscompose}
\newcommand*\hscompose[2]{#1}

\AtHaskellReset{\global\let\hsdot=\hscompose}

% In the beginning, we should reset Haskell once.
\HaskellReset

\makeatother
\EndFmtInput


\renewcommand\Varid[1]{\codefont{#1}}
\let\Conid\Varid

\begin{document}
\maketitle


\section{Introduction}

This is an overview of the techniques described in the paper \citetalias{embedding}.
The paper demonstrates a series of techniques that can be used when folding Domain Specific Languges.
It does so through the use of a simple parallel prefix circuit language~\cite{scans}.


In this overview a small parser combinator language will be used.
This language brings one key feature that was not described in the paper: how to apply these techniques to a typed language.
Only a minimal functionally complete set of combinators have been included in the language to keep it simple.
However, all other combinators usually found in a combinator language can be contructed from this set.


%% The paper takes an example Domain Specific Language and then demonstrates a series of techniques that are helpful for folding them.





%% What is the paper? What are its core ideas?

%% How will the techniques be demoed? Pros/Cons of why they are useful for the parsing language.


\section{Background}

\subsection{DSLs}

A Domain Specific Language (DSL) is a programming language that has a specialised domain or use-case.
This differs from a General Purpose Language (GPL), which can be applied across a larger set of domains.
DSLs can be split into two different categories: standalone and embedded. Standalone DSLs require their own compiler and typically have their own syntax.
Embedded DSLs use a GPL as a host language, therefore they use the syntax and compiler from that GPL.
This means that they are easier to maintain and are often quicker to develop than standalone DSLs.

An embedded DSL can be implemented with two main techniques.
Firstly, a deep approach can be taken, this means that terms in the DSL will construct an Abstract Syntax Tree (AST) as a host language datatype.
This can then be used to apply optimisations and then evaluated.
A second approach is to define the terms as first class components of the language, avoiding the creation of an AST - this is known as a shallow embedding.


\subsection{Parsers}

A parser is a used to convert a series of tokens into another language.
For example converting a string into a Haskell datatype.
Parser combinators provide a flexible approach to constructing parsers.
Unlike parser generators, a combinator library is embedded within a host language: using combinators to construct the grammar.
This makes it a suitable to demonstrate the techniques descriped in this paper for folding the DSL to create parsers.

The langauge is made up of 6 terms, they provide all the essential operations needed in a parser.

\begin{hscode}\SaveRestoreHook
\column{B}{@{}>{\hspre}l<{\hspost}@{}}%
\column{3}{@{}>{\hspre}l<{\hspost}@{}}%
\column{12}{@{}>{\hspre}l<{\hspost}@{}}%
\column{32}{@{}>{\hspre}c<{\hspost}@{}}%
\column{32E}{@{}l@{}}%
\column{36}{@{}>{\hspre}l<{\hspost}@{}}%
\column{46}{@{}>{\hspre}c<{\hspost}@{}}%
\column{46E}{@{}l@{}}%
\column{50}{@{}>{\hspre}l<{\hspost}@{}}%
\column{E}{@{}>{\hspre}l<{\hspost}@{}}%
\>[3]{}\Varid{empty}{}\<[12]%
\>[12]{}\mathbin{::}\Conid{Parser}\codeskip \Varid{a}{}\<[E]%
\\
\>[3]{}\Varid{pure}{}\<[12]%
\>[12]{}\mathbin{::}\Varid{a}{}\<[32]%
\>[32]{}\to {}\<[32E]%
\>[36]{}\Conid{Parser}\codeskip \Varid{a}{}\<[E]%
\\
\>[3]{}\Varid{satisfy}{}\<[12]%
\>[12]{}\mathbin{::}(\Conid{Char}\to \Conid{Bool}){}\<[32]%
\>[32]{}\to {}\<[32E]%
\>[36]{}\Conid{Parser}\codeskip \Conid{Char}{}\<[E]%
\\
\>[3]{}\Varid{try}{}\<[12]%
\>[12]{}\mathbin{::}\Conid{Parser}\codeskip \Varid{a}{}\<[32]%
\>[32]{}\to {}\<[32E]%
\>[36]{}\Conid{Parser}\codeskip \Varid{a}{}\<[E]%
\\
\>[3]{}\Varid{ap}{}\<[12]%
\>[12]{}\mathbin{::}\Conid{Parser}\codeskip (\Varid{a}\to \Varid{b}){}\<[32]%
\>[32]{}\to {}\<[32E]%
\>[36]{}\Conid{Parser}\codeskip \Varid{a}{}\<[46]%
\>[46]{}\to {}\<[46E]%
\>[50]{}\Conid{Parser}\codeskip \Varid{b}{}\<[E]%
\\
\>[3]{}\Varid{or}{}\<[12]%
\>[12]{}\mathbin{::}\Conid{Parser}\codeskip \Varid{a}{}\<[32]%
\>[32]{}\to {}\<[32E]%
\>[36]{}\Conid{Parser}\codeskip \Varid{a}{}\<[46]%
\>[46]{}\to {}\<[46E]%
\>[50]{}\Conid{Parser}\codeskip \Varid{a}{}\<[E]%
\ColumnHook
\end{hscode}\resethooks


For example, a parser that can parse the characters \ensuremath{\text{\ttfamily{\textquotesingle}a\textquotesingle}} or \ensuremath{\text{\ttfamily{\textquotesingle}b\textquotesingle}} can be defined as,
\begin{hscode}\SaveRestoreHook
\column{B}{@{}>{\hspre}l<{\hspost}@{}}%
\column{3}{@{}>{\hspre}l<{\hspost}@{}}%
\column{E}{@{}>{\hspre}l<{\hspost}@{}}%
\>[3]{}\Varid{aorb}\mathbin{::}\Conid{Parser}\codeskip \Conid{Char}{}\<[E]%
\\
\>[3]{}\Varid{aorb}\mathrel{=}\Varid{satisfy}\codeskip (\equiv \text{\ttfamily{\textquotesingle}a\textquotesingle})\mathbin{\text{\`{}}\Varid{or}\text{\`{}}}\Varid{satisfy}\codeskip (\equiv \text{\ttfamily{\textquotesingle}b\textquotesingle}){}\<[E]%
\ColumnHook
\end{hscode}\resethooks

A deep embedding of this parser language is defined in the alegebraic datatype:

\begin{hscode}\SaveRestoreHook
\column{B}{@{}>{\hspre}l<{\hspost}@{}}%
\column{3}{@{}>{\hspre}l<{\hspost}@{}}%
\column{5}{@{}>{\hspre}l<{\hspost}@{}}%
\column{15}{@{}>{\hspre}c<{\hspost}@{}}%
\column{15E}{@{}l@{}}%
\column{19}{@{}>{\hspre}l<{\hspost}@{}}%
\column{37}{@{}>{\hspre}c<{\hspost}@{}}%
\column{37E}{@{}l@{}}%
\column{41}{@{}>{\hspre}l<{\hspost}@{}}%
\column{52}{@{}>{\hspre}c<{\hspost}@{}}%
\column{52E}{@{}l@{}}%
\column{56}{@{}>{\hspre}l<{\hspost}@{}}%
\column{E}{@{}>{\hspre}l<{\hspost}@{}}%
\>[3]{}\mathbf{data}\codeskip \Conid{Parser}_{2}\codeskip (\Varid{a}\mathbin{::}\mathbin{*})\codeskip \mathbf{where}{}\<[E]%
\\
\>[3]{}\hsindent{2}{}\<[5]%
\>[5]{}\Conid{Empty}_{2}{}\<[15]%
\>[15]{}\mathbin{::}{}\<[15E]%
\>[19]{}\Conid{Parser}_{2}\codeskip \Varid{a}{}\<[E]%
\\
\>[3]{}\hsindent{2}{}\<[5]%
\>[5]{}\Conid{Pure}_{2}{}\<[15]%
\>[15]{}\mathbin{::}{}\<[15E]%
\>[19]{}\Varid{a}{}\<[37]%
\>[37]{}\to {}\<[37E]%
\>[41]{}\Conid{Parser}_{2}\codeskip \Varid{a}{}\<[E]%
\\
\>[3]{}\hsindent{2}{}\<[5]%
\>[5]{}\Conid{Satisfy}_{2}{}\<[15]%
\>[15]{}\mathbin{::}{}\<[15E]%
\>[19]{}(\Conid{Char}\to \Conid{Bool}){}\<[37]%
\>[37]{}\to {}\<[37E]%
\>[41]{}\Conid{Parser}_{2}\codeskip \Conid{Char}{}\<[E]%
\\
\>[3]{}\hsindent{2}{}\<[5]%
\>[5]{}\Conid{Try}_{2}{}\<[15]%
\>[15]{}\mathbin{::}{}\<[15E]%
\>[19]{}\Conid{Parser}_{2}\codeskip \Varid{a}{}\<[37]%
\>[37]{}\to {}\<[37E]%
\>[41]{}\Conid{Parser}_{2}\codeskip \Varid{a}{}\<[E]%
\\
\>[3]{}\hsindent{2}{}\<[5]%
\>[5]{}\Conid{Ap}_{2}{}\<[15]%
\>[15]{}\mathbin{::}{}\<[15E]%
\>[19]{}\Conid{Parser}_{2}\codeskip (\Varid{a}\to \Varid{b}){}\<[37]%
\>[37]{}\to {}\<[37E]%
\>[41]{}\Conid{Parser}_{2}\codeskip \Varid{a}{}\<[52]%
\>[52]{}\to {}\<[52E]%
\>[56]{}\Conid{Parser}_{2}\codeskip \Varid{b}{}\<[E]%
\\
\>[3]{}\hsindent{2}{}\<[5]%
\>[5]{}\Conid{Or}_{2}{}\<[15]%
\>[15]{}\mathbin{::}{}\<[15E]%
\>[19]{}\Conid{Parser}_{2}\codeskip \Varid{a}{}\<[37]%
\>[37]{}\to {}\<[37E]%
\>[41]{}\Conid{Parser}_{2}\codeskip \Varid{a}{}\<[52]%
\>[52]{}\to {}\<[52E]%
\>[56]{}\Conid{Parser}_{2}\codeskip \Varid{a}{}\<[E]%
\ColumnHook
\end{hscode}\resethooks

This can be interpretted by defining a function such as \ensuremath{\Varid{size}}, that finds the size of the AST used to construct the parser - this can be found in the appendix.
\ensuremath{\Varid{size}} interprets the deep embedding, by folding over the datatype.
See the appendix for how to add an interpretation with a shallow embedding.


\section{Folds}

It is possible to capture the shape of an abstract datatype through the \ensuremath{\Conid{Functor}} type class.
It is possible to capture the shape of an abstract datatype as a \ensuremath{\Conid{Functor}}.
The use of a \ensuremath{\Conid{Functor}} allows for the specification of where a datatype recurses.
There is however one problem, a functor expresing the parser language is required to be typed.
Parsers require the type of the tokens being parsed.
For example a parser reading tokens that make up an expression could have the type \ensuremath{\Conid{Parser}\codeskip \Conid{Expr}}.
A functor does not retain the type of the parser, therefore it is required to define a special type class called \ensuremath{\Conid{IFunctor}}, which is able to maintain the type indicies~\cite{mcbride2011functional}.
A full definition can be found in the appendix.

The shape of \ensuremath{\Conid{Parser}_{2}}, can be seen in \ensuremath{\Conid{ParserF}} where the \ensuremath{\Varid{k}\codeskip \Varid{a}} marks the recursive spots.

\begin{hscode}\SaveRestoreHook
\column{B}{@{}>{\hspre}l<{\hspost}@{}}%
\column{3}{@{}>{\hspre}l<{\hspost}@{}}%
\column{5}{@{}>{\hspre}l<{\hspost}@{}}%
\column{14}{@{}>{\hspre}l<{\hspost}@{}}%
\column{33}{@{}>{\hspre}l<{\hspost}@{}}%
\column{42}{@{}>{\hspre}l<{\hspost}@{}}%
\column{E}{@{}>{\hspre}l<{\hspost}@{}}%
\>[3]{}\mathbf{data}\codeskip \Conid{ParserF}\codeskip (\Varid{k}\mathbin{::}\mathbin{*}\to \mathbin{*})\codeskip (\Varid{a}\mathbin{::}\mathbin{*})\codeskip \mathbf{where}{}\<[E]%
\\
\>[3]{}\hsindent{2}{}\<[5]%
\>[5]{}\Conid{EmptyF}{}\<[14]%
\>[14]{}\mathbin{::}\Conid{ParserF}\codeskip \Varid{k}\codeskip \Varid{a}{}\<[E]%
\\
\>[3]{}\hsindent{2}{}\<[5]%
\>[5]{}\Conid{PureF}{}\<[14]%
\>[14]{}\mathbin{::}\Varid{a}{}\<[33]%
\>[33]{}\to \Conid{ParserF}\codeskip \Varid{k}\codeskip \Varid{a}{}\<[E]%
\\
\>[3]{}\hsindent{2}{}\<[5]%
\>[5]{}\Conid{SatisfyF}\mathbin{::}(\Conid{Char}\to \Conid{Bool}){}\<[33]%
\>[33]{}\to \Conid{ParserF}\codeskip \Varid{k}\codeskip \Conid{Char}{}\<[E]%
\\
\>[3]{}\hsindent{2}{}\<[5]%
\>[5]{}\Conid{TryF}{}\<[14]%
\>[14]{}\mathbin{::}\Varid{k}\codeskip \Varid{a}{}\<[33]%
\>[33]{}\to \Conid{ParserF}\codeskip \Varid{k}\codeskip \Varid{a}{}\<[E]%
\\
\>[3]{}\hsindent{2}{}\<[5]%
\>[5]{}\Conid{ApF}{}\<[14]%
\>[14]{}\mathbin{::}\Varid{k}\codeskip (\Varid{a}\to \Varid{b}){}\<[33]%
\>[33]{}\to \Varid{k}\codeskip \Varid{a}{}\<[42]%
\>[42]{}\to \Conid{ParserF}\codeskip \Varid{k}\codeskip \Varid{b}{}\<[E]%
\\
\>[3]{}\hsindent{2}{}\<[5]%
\>[5]{}\Conid{OrF}{}\<[14]%
\>[14]{}\mathbin{::}\Varid{k}\codeskip \Varid{a}{}\<[33]%
\>[33]{}\to \Varid{k}\codeskip \Varid{a}{}\<[42]%
\>[42]{}\to \Conid{ParserF}\codeskip \Varid{k}\codeskip \Varid{a}{}\<[E]%
\ColumnHook
\end{hscode}\resethooks
\begin{hscode}\SaveRestoreHook
\column{B}{@{}>{\hspre}l<{\hspost}@{}}%
\column{3}{@{}>{\hspre}l<{\hspost}@{}}%
\column{5}{@{}>{\hspre}l<{\hspost}@{}}%
\column{26}{@{}>{\hspre}l<{\hspost}@{}}%
\column{34}{@{}>{\hspre}l<{\hspost}@{}}%
\column{38}{@{}>{\hspre}l<{\hspost}@{}}%
\column{42}{@{}>{\hspre}l<{\hspost}@{}}%
\column{E}{@{}>{\hspre}l<{\hspost}@{}}%
\>[3]{}\mathbf{instance}\codeskip \Conid{IFunctor}\codeskip \Conid{ParserF}\codeskip \mathbf{where}{}\<[E]%
\\
\>[3]{}\hsindent{2}{}\<[5]%
\>[5]{}\Varid{imap}\codeskip \anonymous \codeskip \Conid{EmptyF}{}\<[26]%
\>[26]{}\mathrel{=}\Conid{EmptyF}{}\<[E]%
\\
\>[3]{}\hsindent{2}{}\<[5]%
\>[5]{}\Varid{imap}\codeskip \anonymous \codeskip (\Conid{PureF}\codeskip \Varid{x}){}\<[26]%
\>[26]{}\mathrel{=}\Conid{PureF}\codeskip {}\<[38]%
\>[38]{}\Varid{x}{}\<[E]%
\\
\>[3]{}\hsindent{2}{}\<[5]%
\>[5]{}\Varid{imap}\codeskip \anonymous \codeskip (\Conid{SatisfyF}\codeskip \Varid{c}){}\<[26]%
\>[26]{}\mathrel{=}\Conid{SatisfyF}\codeskip {}\<[38]%
\>[38]{}\Varid{c}{}\<[E]%
\\
\>[3]{}\hsindent{2}{}\<[5]%
\>[5]{}\Varid{imap}\codeskip \Varid{f}\codeskip (\Conid{TryF}\codeskip \Varid{px}){}\<[26]%
\>[26]{}\mathrel{=}\Conid{TryF}\codeskip {}\<[34]%
\>[34]{}(\Varid{f}\codeskip \Varid{px}){}\<[E]%
\\
\>[3]{}\hsindent{2}{}\<[5]%
\>[5]{}\Varid{imap}\codeskip \Varid{f}\codeskip (\Conid{ApF}\codeskip \Varid{pf}\codeskip \Varid{px}){}\<[26]%
\>[26]{}\mathrel{=}\Conid{ApF}\codeskip {}\<[34]%
\>[34]{}(\Varid{f}\codeskip \Varid{pf})\codeskip {}\<[42]%
\>[42]{}(\Varid{f}\codeskip \Varid{px}){}\<[E]%
\\
\>[3]{}\hsindent{2}{}\<[5]%
\>[5]{}\Varid{imap}\codeskip \Varid{f}\codeskip (\Conid{OrF}\codeskip \Varid{px}\codeskip \Varid{py}){}\<[26]%
\>[26]{}\mathrel{=}\Conid{OrF}\codeskip {}\<[34]%
\>[34]{}(\Varid{f}\codeskip \Varid{px})\codeskip {}\<[42]%
\>[42]{}(\Varid{f}\codeskip \Varid{py}){}\<[E]%
\ColumnHook
\end{hscode}\resethooks

\ensuremath{\Conid{Fix}} is used to get the fixed point of the functor.
It contains the structure needed to make the datatype recursive.
\ensuremath{\Conid{Parser}_{4}} is the fixed point of \ensuremath{\Conid{ParserF}}.
\begin{hscode}\SaveRestoreHook
\column{B}{@{}>{\hspre}l<{\hspost}@{}}%
\column{3}{@{}>{\hspre}l<{\hspost}@{}}%
\column{E}{@{}>{\hspre}l<{\hspost}@{}}%
\>[3]{}\mathbf{type}\codeskip \Conid{Parser}_{4}\codeskip \Varid{a}\mathrel{=}\Conid{Fix}\codeskip \Conid{ParserF}\codeskip \Varid{a}{}\<[E]%
\ColumnHook
\end{hscode}\resethooks
A mechanism is now required for folding this abstract datatype.
This is possible through the use of a catamorphism, which is a generalised way of folding an abstract datatype.
Therefore, the \ensuremath{\Varid{cata}} function can be used - a definition can be found in the appendix.


Now all the building blocks have been defined that allow for the folding of the parser DSL.
\ensuremath{\Varid{size}} can be defined as a fold, which is determined by the \ensuremath{\Varid{sizeAlg}}.
Due to parsers being a typed language, a constant functor is required to preserve the type indicies.

\begin{hscode}\SaveRestoreHook
\column{B}{@{}>{\hspre}l<{\hspost}@{}}%
\column{3}{@{}>{\hspre}l<{\hspost}@{}}%
\column{22}{@{}>{\hspre}l<{\hspost}@{}}%
\column{26}{@{}>{\hspre}l<{\hspost}@{}}%
\column{E}{@{}>{\hspre}l<{\hspost}@{}}%
\>[3]{}\mathbf{type}\codeskip \Conid{ParserAlg}\codeskip \Varid{a}\codeskip \Varid{i}\mathrel{=}\Conid{ParserF}\codeskip \Varid{a}\codeskip \Varid{i}\to \Varid{a}\codeskip \Varid{i}{}\<[E]%
\\[\blanklineskip]%
\>[3]{}\mathbf{newtype}\codeskip \Conid{C}\codeskip \Varid{a}\codeskip \Varid{i}\mathrel{=}\Conid{C}\codeskip \{\mskip1.5mu \Varid{unConst}\mathbin{::}\Varid{a}\mskip1.5mu\}{}\<[E]%
\\[\blanklineskip]%
\>[3]{}\Varid{sizeAlg}\mathbin{::}\Conid{ParserAlg}\codeskip (\Conid{C}\codeskip \Conid{Size})\codeskip \Varid{i}{}\<[E]%
\\
\>[3]{}\Varid{sizeAlg}\codeskip \Conid{EmptyF}{}\<[26]%
\>[26]{}\mathrel{=}\Conid{C}\codeskip \mathrm{1}{}\<[E]%
\\
\>[3]{}\Varid{sizeAlg}\codeskip (\Conid{PureF}\codeskip {}\<[22]%
\>[22]{}\anonymous ){}\<[26]%
\>[26]{}\mathrel{=}\Conid{C}\codeskip \mathrm{1}{}\<[E]%
\\
\>[3]{}\Varid{sizeAlg}\codeskip (\Conid{SatisfyF}\codeskip {}\<[22]%
\>[22]{}\anonymous ){}\<[26]%
\>[26]{}\mathrel{=}\Conid{C}\codeskip \mathrm{1}{}\<[E]%
\\
\>[3]{}\Varid{sizeAlg}\codeskip (\Conid{TryF}\codeskip (\Conid{C}\codeskip \Varid{n})){}\<[26]%
\>[26]{}\mathrel{=}\Conid{C}\mathbin{\$}\Varid{n}\mathbin{+}\mathrm{1}{}\<[E]%
\\
\>[3]{}\Varid{sizeAlg}\codeskip (\Conid{ApF}\codeskip (\Conid{C}\codeskip \Varid{pf})\codeskip (\Conid{C}\codeskip \Varid{px}))\mathrel{=}\Conid{C}\mathbin{\$}\Varid{pf}\mathbin{+}\Varid{px}\mathbin{+}\mathrm{1}{}\<[E]%
\\
\>[3]{}\Varid{sizeAlg}\codeskip (\Conid{OrF}\codeskip (\Conid{C}\codeskip \Varid{px})\codeskip (\Conid{C}\codeskip \Varid{py}))\mathrel{=}\Conid{C}\mathbin{\$}\Varid{px}\mathbin{+}\Varid{py}\mathbin{+}\mathrm{1}{}\<[E]%
\\[\blanklineskip]%
\>[3]{}\Varid{size}_{4}\mathbin{::}\Conid{Parser}_{4}\codeskip \Varid{a}\to \Conid{Size}{}\<[E]%
\\
\>[3]{}\Varid{size}_{4}\mathrel{=}\Varid{unConst}\hsdot{\cdot }{.}\Varid{cata}\codeskip \Varid{sizeAlg}{}\<[E]%
\ColumnHook
\end{hscode}\resethooks

\subsection{Multiple Interpretations}

In DSLs it is common to want to evaluate multiple interpretations.
For example, a parser may also want to know the maximum characters it will read (maximum munch).
In a deep embedding this is simple, a second algebra can be defined.

\begin{hscode}\SaveRestoreHook
\column{B}{@{}>{\hspre}l<{\hspost}@{}}%
\column{3}{@{}>{\hspre}l<{\hspost}@{}}%
\column{10}{@{}>{\hspre}l<{\hspost}@{}}%
\column{29}{@{}>{\hspre}l<{\hspost}@{}}%
\column{E}{@{}>{\hspre}l<{\hspost}@{}}%
\>[3]{}\mathbf{type}\codeskip \Conid{MM}\mathrel{=}\Conid{Int}{}\<[E]%
\\[\blanklineskip]%
\>[3]{}\Varid{mmAlg}\mathbin{::}\Conid{ParserAlg}\codeskip (\Conid{C}\codeskip \Conid{MM})\codeskip \Varid{i}{}\<[E]%
\\
\>[3]{}\Varid{mmAlg}\codeskip (\Conid{PureF}\codeskip \anonymous ){}\<[29]%
\>[29]{}\mathrel{=}\Conid{C}\codeskip \mathrm{0}{}\<[E]%
\\
\>[3]{}\Varid{mmAlg}\codeskip {}\<[10]%
\>[10]{}\Conid{EmptyF}{}\<[29]%
\>[29]{}\mathrel{=}\Conid{C}\codeskip \mathrm{0}{}\<[E]%
\\
\>[3]{}\Varid{mmAlg}\codeskip (\Conid{SatisfyF}\codeskip \Varid{c}){}\<[29]%
\>[29]{}\mathrel{=}\Conid{C}\codeskip \mathrm{1}{}\<[E]%
\\
\>[3]{}\Varid{mmAlg}\codeskip (\Conid{TryF}\codeskip (\Conid{C}\codeskip \Varid{px})){}\<[29]%
\>[29]{}\mathrel{=}\Conid{C}\codeskip \Varid{px}{}\<[E]%
\\
\>[3]{}\Varid{mmAlg}\codeskip (\Conid{ApF}\codeskip (\Conid{C}\codeskip \Varid{pf})\codeskip (\Conid{C}\codeskip \Varid{px}))\mathrel{=}\Conid{C}\mathbin{\$}\Varid{pf}\mathbin{+}\Varid{px}{}\<[E]%
\\
\>[3]{}\Varid{mmAlg}\codeskip (\Conid{OrF}\codeskip (\Conid{C}\codeskip \Varid{px})\codeskip (\Conid{C}\codeskip \Varid{py}))\mathrel{=}\Conid{C}\mathbin{\$}\Varid{max}\codeskip \Varid{px}\codeskip \Varid{py}{}\<[E]%
\\[\blanklineskip]%
\>[3]{}\Varid{maxMunch}_{4}\mathbin{::}\Conid{Parser}_{4}\codeskip \Varid{a}\to \Conid{MM}{}\<[E]%
\\
\>[3]{}\Varid{maxMunch}_{4}\mathrel{=}\Varid{unConst}\hsdot{\cdot }{.}\Varid{cata}\codeskip \Varid{mmAlg}{}\<[E]%
\ColumnHook
\end{hscode}\resethooks
However, in a shallow embedding it is not as easy.
To be able to evaluate both semantics a pair can be used, with both interpretations being evaluated simultaneously.
If many semantics are required this can become cumbersome to define.

\begin{hscode}\SaveRestoreHook
\column{B}{@{}>{\hspre}l<{\hspost}@{}}%
\column{3}{@{}>{\hspre}l<{\hspost}@{}}%
\column{5}{@{}>{\hspre}l<{\hspost}@{}}%
\column{16}{@{}>{\hspre}l<{\hspost}@{}}%
\column{30}{@{}>{\hspre}l<{\hspost}@{}}%
\column{43}{@{}>{\hspre}l<{\hspost}@{}}%
\column{E}{@{}>{\hspre}l<{\hspost}@{}}%
\>[3]{}\mathbf{type}\codeskip \Conid{Parser}_{5}\mathrel{=}(\Conid{Size},\Conid{MM}){}\<[E]%
\\[\blanklineskip]%
\>[3]{}\Varid{size}_{5}\mathbin{::}\Conid{Parser}_{5}\to \Conid{Size}{}\<[E]%
\\
\>[3]{}\Varid{size}_{5}\mathrel{=}\Varid{fst}{}\<[E]%
\\[\blanklineskip]%
\>[3]{}\Varid{maxMunch}_{5}\mathbin{::}\Conid{Parser}_{5}\to \Conid{Size}{}\<[E]%
\\
\>[3]{}\Varid{maxMunch}_{5}\mathrel{=}\Varid{snd}{}\<[E]%
\\[\blanklineskip]%
\>[3]{}\Varid{smmAlg}\mathbin{::}\Conid{ParserAlg}\codeskip (\Conid{C}\codeskip (\Conid{Size},\Conid{MM}))\codeskip \Varid{a}{}\<[E]%
\\
\>[3]{}\Varid{smmAlg}\codeskip (\Conid{PureF}\codeskip \anonymous ){}\<[30]%
\>[30]{}\mathrel{=}\Conid{C}\codeskip (\mathrm{1},{}\<[43]%
\>[43]{}\mathrm{0}){}\<[E]%
\\
\>[3]{}\Varid{smmAlg}\codeskip \Conid{EmptyF}{}\<[30]%
\>[30]{}\mathrel{=}\Conid{C}\codeskip (\mathrm{1},{}\<[43]%
\>[43]{}\mathrm{0}){}\<[E]%
\\
\>[3]{}\Varid{smmAlg}\codeskip (\Conid{SatisfyF}\codeskip \Varid{c}){}\<[30]%
\>[30]{}\mathrel{=}\Conid{C}\codeskip (\mathrm{1},{}\<[43]%
\>[43]{}\mathrm{1}){}\<[E]%
\\
\>[3]{}\Varid{smmAlg}\codeskip (\Conid{TryF}\codeskip (\Conid{C}\codeskip (\Varid{s},\Varid{mm}))){}\<[30]%
\>[30]{}\mathrel{=}\Conid{C}\codeskip (\Varid{s}\mathbin{+}\mathrm{1},{}\<[43]%
\>[43]{}\Varid{mm}){}\<[E]%
\\
\>[3]{}\Varid{smmAlg}\codeskip (\Conid{ApF}\codeskip {}\<[16]%
\>[16]{}(\Conid{C}\codeskip (\Varid{s},\Varid{mm}))\codeskip (\Conid{C}\codeskip (\Varid{s}',\Varid{mm'}))){}\<[E]%
\\
\>[3]{}\hsindent{2}{}\<[5]%
\>[5]{}\mathrel{=}\Conid{C}\codeskip (\Varid{s}\mathbin{+}\Varid{s}'\mathbin{+}\mathrm{1},\Varid{mm}\mathbin{+}\Varid{mm'}){}\<[E]%
\\
\>[3]{}\Varid{smmAlg}\codeskip (\Conid{OrF}\codeskip {}\<[16]%
\>[16]{}(\Conid{C}\codeskip (\Varid{s},\Varid{mm}))\codeskip (\Conid{C}\codeskip (\Varid{s}',\Varid{mm'}))){}\<[E]%
\\
\>[3]{}\hsindent{2}{}\<[5]%
\>[5]{}\mathrel{=}\Conid{C}\codeskip (\Varid{s}\mathbin{+}\Varid{s}'\mathbin{+}\mathrm{1},\Varid{max}\codeskip \Varid{mm}\codeskip \Varid{mm'}){}\<[E]%
\ColumnHook
\end{hscode}\resethooks

Although this is an algebra, you are able to learn the shallow embedding from this, for example:

\begin{hscode}\SaveRestoreHook
\column{B}{@{}>{\hspre}l<{\hspost}@{}}%
\column{3}{@{}>{\hspre}l<{\hspost}@{}}%
\column{E}{@{}>{\hspre}l<{\hspost}@{}}%
\>[3]{}\Varid{ap}_{5}\codeskip \Varid{pf}\codeskip \Varid{px}\mathrel{=}\Varid{smmAlg}\codeskip (\Conid{ApF}\codeskip \Varid{pf}\codeskip \Varid{px}){}\<[E]%
\\
\>[3]{}\Varid{or}_{5}\codeskip \Varid{px}\codeskip \Varid{py}\mathrel{=}\Varid{smmAlg}\codeskip (\Conid{OrF}\codeskip \Varid{px}\codeskip \Varid{py}){}\<[E]%
\ColumnHook
\end{hscode}\resethooks

\subsection{Dependent Interpretations}

In a more complex parser combinator library that perform optimisations on a deep embedding,
it could also be possible that there is a primary fold that depends on other secondary folds on parts of the AST.
Folds such as this are named mutumorphisms~\cite{Fokkinga1989TuplingAM},
they can be implemented by tupling the functions in the fold.
\citet{parsley} makes use of a zygomorphism -
a special case where the dependency is only one-way - to perform consumption analysis.


\subsection{Context-sensitive Interpretations}


Parsers themselves inherently require context sensitive interpretations - what can be parsed will
depend on what has previously been parsed.

Using the semantics from~\citet{wuYoda}, an implementation can be given for a simple parser using an accumulating fold.

\begin{hscode}\SaveRestoreHook
\column{B}{@{}>{\hspre}l<{\hspost}@{}}%
\column{3}{@{}>{\hspre}l<{\hspost}@{}}%
\column{E}{@{}>{\hspre}l<{\hspost}@{}}%
\>[3]{}\mathbf{newtype}\codeskip \Conid{Y}\codeskip \Varid{a}\mathrel{=}\Conid{Y}\codeskip \{\mskip1.5mu \Varid{unYoda}\mathbin{::}\Conid{String}\to [\mskip1.5mu (\Varid{a},\Conid{String})\mskip1.5mu]\mskip1.5mu\}{}\<[E]%
\ColumnHook
\end{hscode}\resethooks
\begin{hscode}\SaveRestoreHook
\column{B}{@{}>{\hspre}l<{\hspost}@{}}%
\column{3}{@{}>{\hspre}l<{\hspost}@{}}%
\column{5}{@{}>{\hspre}l<{\hspost}@{}}%
\column{9}{@{}>{\hspre}l<{\hspost}@{}}%
\column{14}{@{}>{\hspre}l<{\hspost}@{}}%
\column{19}{@{}>{\hspre}c<{\hspost}@{}}%
\column{19E}{@{}l@{}}%
\column{22}{@{}>{\hspre}l<{\hspost}@{}}%
\column{23}{@{}>{\hspre}l<{\hspost}@{}}%
\column{28}{@{}>{\hspre}l<{\hspost}@{}}%
\column{35}{@{}>{\hspre}l<{\hspost}@{}}%
\column{E}{@{}>{\hspre}l<{\hspost}@{}}%
\>[3]{}\Varid{yAlg}\mathbin{::}\Conid{ParserAlg}\codeskip \Conid{Y}\codeskip \Varid{i}{}\<[E]%
\\
\>[3]{}\Varid{yAlg}\codeskip (\Conid{PureF}\codeskip \Varid{x}){}\<[22]%
\>[22]{}\mathrel{=}\Conid{Y}\mathbin{\$}\lambda \Varid{ts}\to [\mskip1.5mu (\Varid{x},\Varid{ts})\mskip1.5mu]{}\<[E]%
\\
\>[3]{}\Varid{yAlg}\codeskip {}\<[9]%
\>[9]{}\Conid{EmptyF}{}\<[22]%
\>[22]{}\mathrel{=}\Conid{Y}\mathbin{\$}\Varid{const}\codeskip [\mskip1.5mu \mskip1.5mu]{}\<[E]%
\\
\>[3]{}\Varid{yAlg}\codeskip (\Conid{SatisfyF}\codeskip \Varid{c}){}\<[22]%
\>[22]{}\mathrel{=}\Conid{Y}\mathbin{\$}\lambda \mathbf{case}{}\<[E]%
\\
\>[3]{}\hsindent{2}{}\<[5]%
\>[5]{}[\mskip1.5mu \mskip1.5mu]{}\<[14]%
\>[14]{}\to [\mskip1.5mu \mskip1.5mu]{}\<[E]%
\\
\>[3]{}\hsindent{2}{}\<[5]%
\>[5]{}(\Varid{t}\mathbin{:}\Varid{ts}'){}\<[14]%
\>[14]{}\to [\mskip1.5mu (\Varid{t},\Varid{ts}')\mid \Varid{c}\codeskip \Varid{t}\mskip1.5mu]{}\<[E]%
\\
\>[3]{}\Varid{yAlg}\codeskip (\Conid{TryF}\codeskip \Varid{px}){}\<[22]%
\>[22]{}\mathrel{=}\Varid{px}{}\<[E]%
\\
\>[3]{}\Varid{yAlg}\codeskip (\Conid{ApF}\codeskip (\Conid{Y}\codeskip \Varid{pf})\codeskip (\Conid{Y}\codeskip \Varid{px}))\mathrel{=}\Conid{Y}\mathbin{\$}\lambda \Varid{ts}\to {}\<[E]%
\\
\>[3]{}\hsindent{2}{}\<[5]%
\>[5]{}[\mskip1.5mu (\Varid{f}\codeskip \Varid{x},\Varid{ts}''){}\<[19]%
\>[19]{}\mid {}\<[19E]%
\>[23]{}(\Varid{f},{}\<[28]%
\>[28]{}\Varid{ts}'){}\<[35]%
\>[35]{}\leftarrow \Varid{pf}\codeskip \Varid{ts}{}\<[E]%
\\
\>[19]{},{}\<[19E]%
\>[23]{}(\Varid{x},{}\<[28]%
\>[28]{}\Varid{ts}''){}\<[35]%
\>[35]{}\leftarrow \Varid{px}\codeskip \Varid{ts}'\mskip1.5mu]{}\<[E]%
\\
\>[3]{}\Varid{yAlg}\codeskip (\Conid{OrF}\codeskip (\Conid{Y}\codeskip \Varid{px})\codeskip (\Conid{Y}\codeskip \Varid{py}))\mathrel{=}\Conid{Y}\mathbin{\$}\lambda \Varid{ts}\to \Varid{px}\codeskip \Varid{ts}\plus \Varid{py}\codeskip \Varid{ts}{}\<[E]%
\\[\blanklineskip]%
\>[3]{}\Varid{parse}\mathbin{::}\Conid{Parser}_{4}\codeskip \Varid{a}\to (\Conid{String}\to [\mskip1.5mu (\Varid{a},\Conid{String})\mskip1.5mu]){}\<[E]%
\\
\>[3]{}\Varid{parse}\mathrel{=}\Varid{unYoda}\hsdot{\cdot }{.}\Varid{cata}\codeskip \Varid{yAlg}{}\<[E]%
\ColumnHook
\end{hscode}\resethooks

\subsection{Parameterized Interpretations}


Previously, when defining multiple interpretations in a shallow embedding, a tuple was used.
However, this does not extend well when many interpretations are needed.
Large tuples tend to lack good language support and will become messy to work with.
It would be beneficial if a shallow embedding could have a parameter that gives it the interpretation.

\ensuremath{\Conid{Parser}_{7}} allows for this approach,
the shallow embedding is made up of first class functions that require an algebra argument.
This algebra describes how the shallow embedding should fold the structure.

\begin{hscode}\SaveRestoreHook
\column{B}{@{}>{\hspre}l<{\hspost}@{}}%
\column{3}{@{}>{\hspre}l<{\hspost}@{}}%
\column{5}{@{}>{\hspre}l<{\hspost}@{}}%
\column{E}{@{}>{\hspre}l<{\hspost}@{}}%
\>[3]{}\mathbf{newtype}\codeskip \Conid{Parser}_{7}\codeskip \Varid{i}\mathrel{=}\Conid{P}_{7}{}\<[E]%
\\
\>[3]{}\hsindent{2}{}\<[5]%
\>[5]{}\{\mskip1.5mu \Varid{unP}_{7}\mathbin{::}\forall \Varid{a}\hsforall \hsdot{\cdot }{.}(\forall \Varid{j}\hsforall \hsdot{\cdot }{.}\Conid{ParserF}\codeskip \Varid{a}\codeskip \Varid{j}\to \Varid{a}\codeskip \Varid{j})\to \Varid{a}\codeskip \Varid{i}\mskip1.5mu\}{}\<[E]%
\\[\blanklineskip]%
\>[3]{}\Varid{pure}_{7}\mathbin{::}\Varid{a}\to \Conid{Parser}_{7}\codeskip \Varid{a}{}\<[E]%
\\
\>[3]{}\Varid{pure}_{7}\codeskip \Varid{x}\mathrel{=}\Conid{P}_{7}\codeskip (\lambda \Varid{h}\to \Varid{h}\codeskip (\Conid{PureF}\codeskip \Varid{x})){}\<[E]%
\\[\blanklineskip]%
\>[3]{}\Varid{empty}_{7}\mathbin{::}\Conid{Parser}_{7}\codeskip \Varid{a}{}\<[E]%
\\
\>[3]{}\Varid{empty}_{7}\mathrel{=}\Conid{P}_{7}\codeskip (\lambda \Varid{h}\to \Varid{h}\codeskip \Conid{EmptyF}){}\<[E]%
\\[\blanklineskip]%
\>[3]{}\Varid{satisfy}_{7}\mathbin{::}(\Conid{Char}\to \Conid{Bool})\to \Conid{Parser}_{7}\codeskip \Conid{Char}{}\<[E]%
\\
\>[3]{}\Varid{satisfy}_{7}\codeskip \Varid{c}\mathrel{=}\Conid{P}_{7}\codeskip (\lambda \Varid{h}\to \Varid{h}\codeskip (\Conid{SatisfyF}\codeskip \Varid{c})){}\<[E]%
\\[\blanklineskip]%
\>[3]{}\Varid{try}_{7}\mathbin{::}\Conid{Parser}_{7}\codeskip \Varid{a}\to \Conid{Parser}_{7}\codeskip \Varid{a}{}\<[E]%
\\
\>[3]{}\Varid{try}_{7}\codeskip \Varid{px}\mathrel{=}\Conid{P}_{7}\codeskip (\lambda \Varid{h}\to \Varid{h}\codeskip (\Conid{TryF}\codeskip (\Varid{unP}_{7}\codeskip \Varid{px}\codeskip \Varid{h}))){}\<[E]%
\\[\blanklineskip]%
\>[3]{}\Varid{ap}_{7}\mathbin{::}\Conid{Parser}_{7}\codeskip (\Varid{a}\to \Varid{b})\to \Conid{Parser}_{7}\codeskip \Varid{a}\to \Conid{Parser}_{7}\codeskip \Varid{b}{}\<[E]%
\\
\>[3]{}\Varid{ap}_{7}\codeskip \Varid{pf}\codeskip \Varid{px}\mathrel{=}\Conid{P}_{7}\codeskip (\lambda \Varid{h}\to \Varid{h}\codeskip (\Conid{ApF}\codeskip (\Varid{unP}_{7}\codeskip \Varid{pf}\codeskip \Varid{h})\codeskip (\Varid{unP}_{7}\codeskip \Varid{px}\codeskip \Varid{h}))){}\<[E]%
\\[\blanklineskip]%
\>[3]{}\Varid{or}_{7}\mathbin{::}\Conid{Parser}_{7}\codeskip \Varid{a}\to \Conid{Parser}_{7}\codeskip \Varid{a}\to \Conid{Parser}_{7}\codeskip \Varid{a}{}\<[E]%
\\
\>[3]{}\Varid{or}_{7}\codeskip \Varid{px}\codeskip \Varid{py}\mathrel{=}\Conid{P}_{7}\codeskip (\lambda \Varid{h}\to \Varid{h}\codeskip (\Conid{OrF}\codeskip (\Varid{unP}_{7}\codeskip \Varid{px}\codeskip \Varid{h})\codeskip (\Varid{unP}_{7}\codeskip \Varid{py}\codeskip \Varid{h}))){}\<[E]%
\ColumnHook
\end{hscode}\resethooks
One benefit of this approach is that it allows the shallow embedding to be converted to a deep embedding.
\begin{hscode}\SaveRestoreHook
\column{B}{@{}>{\hspre}l<{\hspost}@{}}%
\column{3}{@{}>{\hspre}l<{\hspost}@{}}%
\column{E}{@{}>{\hspre}l<{\hspost}@{}}%
\>[3]{}\Varid{deep}\mathbin{::}\Conid{Parser}_{7}\codeskip \Varid{a}\to \Conid{Parser}_{4}\codeskip \Varid{a}{}\<[E]%
\\
\>[3]{}\Varid{deep}\codeskip \Varid{parser}\mathrel{=}\Varid{unP}_{7}\codeskip \Varid{parser}\codeskip \Conid{In}{}\<[E]%
\ColumnHook
\end{hscode}\resethooks

Simillarly it is possible to convert a deep embedding into a parameterised shallow embedding.
\begin{hscode}\SaveRestoreHook
\column{B}{@{}>{\hspre}l<{\hspost}@{}}%
\column{3}{@{}>{\hspre}l<{\hspost}@{}}%
\column{28}{@{}>{\hspre}l<{\hspost}@{}}%
\column{E}{@{}>{\hspre}l<{\hspost}@{}}%
\>[3]{}\Varid{shallow}\mathbin{::}\Conid{Parser}_{4}\codeskip \Varid{a}\to \Conid{Parser}_{7}\codeskip \Varid{a}{}\<[E]%
\\
\>[3]{}\Varid{shallow}\mathrel{=}\Varid{cata}\codeskip \Varid{shallowAlg}{}\<[E]%
\\[\blanklineskip]%
\>[3]{}\Varid{shallowAlg}\mathbin{::}\Conid{ParserAlg}\codeskip \Conid{Parser}_{7}\codeskip \Varid{i}{}\<[E]%
\\
\>[3]{}\Varid{shallowAlg}\codeskip (\Conid{PureF}\codeskip \Varid{x}){}\<[28]%
\>[28]{}\mathrel{=}\Varid{pure}_{7}\codeskip \Varid{x}{}\<[E]%
\\
\>[3]{}\Varid{shallowAlg}\codeskip \Conid{EmptyF}{}\<[28]%
\>[28]{}\mathrel{=}\Varid{empty}_{7}{}\<[E]%
\\
\>[3]{}\Varid{shallowAlg}\codeskip (\Conid{SatisfyF}\codeskip \Varid{c}){}\<[28]%
\>[28]{}\mathrel{=}\Varid{satisfy}_{7}\codeskip \Varid{c}{}\<[E]%
\\
\>[3]{}\Varid{shallowAlg}\codeskip (\Conid{TryF}\codeskip \Varid{px}){}\<[28]%
\>[28]{}\mathrel{=}\Varid{try}_{7}\codeskip \Varid{px}{}\<[E]%
\\
\>[3]{}\Varid{shallowAlg}\codeskip (\Conid{ApF}\codeskip \Varid{pf}\codeskip \Varid{px}){}\<[28]%
\>[28]{}\mathrel{=}\Varid{ap}_{7}\codeskip \Varid{pf}\codeskip \Varid{px}{}\<[E]%
\\
\>[3]{}\Varid{shallowAlg}\codeskip (\Conid{OrF}\codeskip \Varid{px}\codeskip \Varid{py}){}\<[28]%
\>[28]{}\mathrel{=}\Varid{or}_{7}\codeskip \Varid{px}\codeskip \Varid{py}{}\<[E]%
\ColumnHook
\end{hscode}\resethooks
Being able to convert between both types of embedding,
demonstrates that deep and parameterised shallow embeddings are inverses of each other.


\subsection{Implicitly Parameterised Interpretations}

The previous parameterised implementation still required the algebra to be specified.
It would be helpful if it could be passed implicitly, if it can be determined from the type of the interpretation.
This is possible in Haskell through the use of a type class.

\begin{hscode}\SaveRestoreHook
\column{B}{@{}>{\hspre}l<{\hspost}@{}}%
\column{3}{@{}>{\hspre}l<{\hspost}@{}}%
\column{5}{@{}>{\hspre}l<{\hspost}@{}}%
\column{15}{@{}>{\hspre}l<{\hspost}@{}}%
\column{35}{@{}>{\hspre}l<{\hspost}@{}}%
\column{E}{@{}>{\hspre}l<{\hspost}@{}}%
\>[3]{}\mathbf{class}\codeskip \Conid{Parser}_{8}\codeskip \Varid{parser}\codeskip \mathbf{where}{}\<[E]%
\\
\>[3]{}\hsindent{2}{}\<[5]%
\>[5]{}\Varid{empty}_{8}{}\<[15]%
\>[15]{}\mathbin{::}\Varid{parser}\codeskip \Varid{a}{}\<[E]%
\\
\>[3]{}\hsindent{2}{}\<[5]%
\>[5]{}\Varid{pure}_{8}{}\<[15]%
\>[15]{}\mathbin{::}\Varid{a}\to \Varid{parser}\codeskip \Varid{a}{}\<[E]%
\\
\>[3]{}\hsindent{2}{}\<[5]%
\>[5]{}\Varid{satisfy}_{8}{}\<[15]%
\>[15]{}\mathbin{::}(\Conid{Char}\to \Conid{Bool}){}\<[35]%
\>[35]{}\to \Varid{parser}\codeskip \Conid{Char}{}\<[E]%
\\
\>[3]{}\hsindent{2}{}\<[5]%
\>[5]{}\Varid{try}_{8}{}\<[15]%
\>[15]{}\mathbin{::}\Varid{parser}\codeskip \Varid{a}{}\<[35]%
\>[35]{}\to \Varid{parser}\codeskip \Varid{a}{}\<[E]%
\\
\>[3]{}\hsindent{2}{}\<[5]%
\>[5]{}\Varid{ap}_{8}{}\<[15]%
\>[15]{}\mathbin{::}\Varid{parser}\codeskip (\Varid{a}\to \Varid{b}){}\<[35]%
\>[35]{}\to \Varid{parser}\codeskip \Varid{a}\to \Varid{parser}\codeskip \Varid{b}{}\<[E]%
\\
\>[3]{}\hsindent{2}{}\<[5]%
\>[5]{}\Varid{or}_{8}{}\<[15]%
\>[15]{}\mathbin{::}\Varid{parser}\codeskip \Varid{a}{}\<[35]%
\>[35]{}\to \Varid{parser}\codeskip \Varid{a}\to \Varid{parser}\codeskip \Varid{a}{}\<[E]%
\ColumnHook
\end{hscode}\resethooks
\begin{hscode}\SaveRestoreHook
\column{B}{@{}>{\hspre}l<{\hspost}@{}}%
\column{3}{@{}>{\hspre}l<{\hspost}@{}}%
\column{E}{@{}>{\hspre}l<{\hspost}@{}}%
\>[3]{}\mathbf{newtype}\codeskip \Conid{Size}_{8}\codeskip \Varid{i}\mathrel{=}\Conid{Size}\codeskip \{\mskip1.5mu \Varid{unSize}\mathbin{::}\Conid{Int}\mskip1.5mu\}\codeskip \mathbf{deriving}\codeskip \Conid{Num}{}\<[E]%
\ColumnHook
\end{hscode}\resethooks
\begin{hscode}\SaveRestoreHook
\column{B}{@{}>{\hspre}l<{\hspost}@{}}%
\column{3}{@{}>{\hspre}l<{\hspost}@{}}%
\column{5}{@{}>{\hspre}l<{\hspost}@{}}%
\column{17}{@{}>{\hspre}l<{\hspost}@{}}%
\column{E}{@{}>{\hspre}l<{\hspost}@{}}%
\>[3]{}\mathbf{instance}\codeskip \Conid{Parser}_{8}\codeskip \Conid{Size}_{8}\codeskip \mathbf{where}{}\<[E]%
\\
\>[3]{}\hsindent{2}{}\<[5]%
\>[5]{}\Varid{empty}_{8}{}\<[17]%
\>[17]{}\mathrel{=}\mathrm{1}{}\<[E]%
\\
\>[3]{}\hsindent{2}{}\<[5]%
\>[5]{}\Varid{pure}_{8}\codeskip \anonymous {}\<[17]%
\>[17]{}\mathrel{=}\mathrm{1}{}\<[E]%
\\
\>[3]{}\hsindent{2}{}\<[5]%
\>[5]{}\Varid{satisfy}_{8}\codeskip \anonymous {}\<[17]%
\>[17]{}\mathrel{=}\mathrm{1}{}\<[E]%
\\
\>[3]{}\hsindent{2}{}\<[5]%
\>[5]{}\Varid{try}_{8}\codeskip \Varid{px}{}\<[17]%
\>[17]{}\mathrel{=}\Varid{px}\mathbin{+}\mathrm{1}{}\<[E]%
\\
\>[3]{}\hsindent{2}{}\<[5]%
\>[5]{}\Varid{ap}_{8}\codeskip \Varid{pf}\codeskip \Varid{px}{}\<[17]%
\>[17]{}\mathrel{=}\Varid{coerce}\codeskip \Varid{pf}\mathbin{+}\Varid{coerce}\codeskip \Varid{px}\mathbin{+}\mathrm{1}{}\<[E]%
\\
\>[3]{}\hsindent{2}{}\<[5]%
\>[5]{}\Varid{or}_{8}\codeskip \Varid{px}\codeskip \Varid{py}{}\<[17]%
\>[17]{}\mathrel{=}\Varid{px}\mathbin{+}\Varid{py}\mathbin{+}\mathrm{1}{}\<[E]%
\ColumnHook
\end{hscode}\resethooks
\ensuremath{\Varid{coerce}} allows for conversion between types that have the same runtime representation.
This is the case for \ensuremath{\Conid{Size}_{8}} and \ensuremath{\Conid{Int}}.
To be able to reuse the previously defined algebras, a different type class can be defined.

\begin{hscode}\SaveRestoreHook
\column{B}{@{}>{\hspre}l<{\hspost}@{}}%
\column{3}{@{}>{\hspre}l<{\hspost}@{}}%
\column{5}{@{}>{\hspre}l<{\hspost}@{}}%
\column{E}{@{}>{\hspre}l<{\hspost}@{}}%
\>[3]{}\mathbf{class}\codeskip \Conid{Parser}_{9}\codeskip \Varid{parser}\codeskip \mathbf{where}{}\<[E]%
\\
\>[3]{}\hsindent{2}{}\<[5]%
\>[5]{}\Varid{alg}\mathbin{::}\Conid{ParserAlg}\codeskip \Varid{parser}\codeskip \Varid{i}{}\<[E]%
\\[\blanklineskip]%
\>[3]{}\mathbf{instance}\codeskip \Conid{Parser}_{9}\codeskip \Conid{Size}_{8}\codeskip \mathbf{where}{}\<[E]%
\\
\>[3]{}\hsindent{2}{}\<[5]%
\>[5]{}\Varid{alg}\mathrel{=}\Varid{coerce}\hsdot{\cdot }{.}\Varid{sizeAlg}\hsdot{\cdot }{.}\Varid{imap}\codeskip \Varid{coerce}{}\<[E]%
\ColumnHook
\end{hscode}\resethooks



\subsection{Modular Interpretations}

There may be times when adding extra combinators would be convenient, for example adding a 'many' operator that allows for
A modular technique to assembling DSLs would aid this process.


\begin{hscode}\SaveRestoreHook
\column{B}{@{}>{\hspre}l<{\hspost}@{}}%
\column{3}{@{}>{\hspre}l<{\hspost}@{}}%
\column{5}{@{}>{\hspre}l<{\hspost}@{}}%
\column{E}{@{}>{\hspre}l<{\hspost}@{}}%
\>[3]{}\mathbf{data}\codeskip \Conid{Empty}_{10}\codeskip (\Varid{k}\mathbin{::}\mathbin{*}\to \mathbin{*})\codeskip (\Varid{a}\mathbin{::}\mathbin{*})\codeskip \mathbf{where}{}\<[E]%
\\
\>[3]{}\hsindent{2}{}\<[5]%
\>[5]{}\Conid{Empty}_{10}\mathbin{::}\Conid{Empty}_{10}\codeskip \Varid{k}\codeskip \Varid{a}{}\<[E]%
\\[\blanklineskip]%
\>[3]{}\mathbf{data}\codeskip \Conid{Pure}_{10}\codeskip (\Varid{k}\mathbin{::}\mathbin{*}\to \mathbin{*})\codeskip (\Varid{a}\mathbin{::}\mathbin{*})\codeskip \mathbf{where}{}\<[E]%
\\
\>[3]{}\hsindent{2}{}\<[5]%
\>[5]{}\Conid{Pure}_{10}\mathbin{::}\Varid{a}\to \Conid{Pure}_{10}\codeskip \Varid{k}\codeskip \Varid{a}{}\<[E]%
\\[\blanklineskip]%
\>[3]{}\mathbf{data}\codeskip \Conid{Satisfy}_{10}\codeskip (\Varid{k}\mathbin{::}\mathbin{*}\to \mathbin{*})\codeskip (\Varid{a}\mathbin{::}\mathbin{*})\codeskip \mathbf{where}{}\<[E]%
\\
\>[3]{}\hsindent{2}{}\<[5]%
\>[5]{}\Conid{Satisfy}_{10}\mathbin{::}(\Conid{Char}\to \Conid{Bool})\to \Conid{Satisfy}_{10}\codeskip \Varid{k}\codeskip \Conid{Char}{}\<[E]%
\\[\blanklineskip]%
\>[3]{}\mathbf{data}\codeskip \Conid{Try}_{10}\codeskip (\Varid{k}\mathbin{::}\mathbin{*}\to \mathbin{*})\codeskip (\Varid{a}\mathbin{::}\mathbin{*})\codeskip \mathbf{where}{}\<[E]%
\\
\>[3]{}\hsindent{2}{}\<[5]%
\>[5]{}\Conid{Try}_{10}\mathbin{::}\Varid{k}\codeskip \Varid{a}\to \Conid{Try}_{10}\codeskip \Varid{k}\codeskip \Varid{a}{}\<[E]%
\\[\blanklineskip]%
\>[3]{}\mathbf{data}\codeskip \Conid{Ap}_{10}\codeskip (\Varid{k}\mathbin{::}\mathbin{*}\to \mathbin{*})\codeskip (\Varid{a}\mathbin{::}\mathbin{*})\codeskip \mathbf{where}{}\<[E]%
\\
\>[3]{}\hsindent{2}{}\<[5]%
\>[5]{}\Conid{Ap}_{10}\mathbin{::}\Varid{k}\codeskip (\Varid{a}\to \Varid{b})\to \Varid{k}\codeskip \Varid{a}\to \Conid{Ap}_{10}\codeskip \Varid{k}\codeskip \Varid{b}{}\<[E]%
\\[\blanklineskip]%
\>[3]{}\mathbf{data}\codeskip \Conid{Or}_{10}\codeskip (\Varid{k}\mathbin{::}\mathbin{*}\to \mathbin{*})\codeskip (\Varid{a}\mathbin{::}\mathbin{*})\codeskip \mathbf{where}{}\<[E]%
\\
\>[3]{}\hsindent{2}{}\<[5]%
\>[5]{}\Conid{Or}_{10}\mathbin{::}\Varid{k}\codeskip \Varid{a}\to \Varid{k}\codeskip \Varid{a}\to \Conid{Or}_{10}\codeskip \Varid{k}\codeskip \Varid{a}{}\<[E]%
\ColumnHook
\end{hscode}\resethooks
\begin{hscode}\SaveRestoreHook
\column{B}{@{}>{\hspre}l<{\hspost}@{}}%
\column{3}{@{}>{\hspre}l<{\hspost}@{}}%
\column{5}{@{}>{\hspre}l<{\hspost}@{}}%
\column{7}{@{}>{\hspre}l<{\hspost}@{}}%
\column{E}{@{}>{\hspre}l<{\hspost}@{}}%
\>[3]{}\mathbf{data}\codeskip (\Varid{f}:\!\!+\!\!:\Varid{g})\codeskip (\Varid{k}\mathbin{::}\mathbin{*}\to \mathbin{*})\codeskip (\Varid{a}\mathbin{::}\mathbin{*})\codeskip \mathbf{where}{}\<[E]%
\\
\>[3]{}\hsindent{2}{}\<[5]%
\>[5]{}\Conid{L}\mathbin{::}\Varid{f}\codeskip \Varid{k}\codeskip \Varid{a}\to (\Varid{f}:\!\!+\!\!:\Varid{g})\codeskip \Varid{k}\codeskip \Varid{a}{}\<[E]%
\\
\>[3]{}\hsindent{2}{}\<[5]%
\>[5]{}\Conid{R}\mathbin{::}\Varid{g}\codeskip \Varid{k}\codeskip \Varid{a}\to (\Varid{f}:\!\!+\!\!:\Varid{g})\codeskip \Varid{k}\codeskip \Varid{a}{}\<[E]%
\\
\>[3]{}\mathbf{infixr}:\!\!+\!\!:{}\<[E]%
\\[\blanklineskip]%
\>[3]{}\mathbf{instance}\codeskip (\Conid{IFunctor}\codeskip \Varid{f},\Conid{IFunctor}\codeskip \Varid{g}){}\<[E]%
\\
\>[3]{}\hsindent{4}{}\<[7]%
\>[7]{}\Rightarrow \Conid{IFunctor}\codeskip (\Varid{f}:\!\!+\!\!:\Varid{g})\codeskip \mathbf{where}{}\<[E]%
\\
\>[3]{}\hsindent{2}{}\<[5]%
\>[5]{}\Varid{imap}\codeskip \Varid{f}\codeskip (\Conid{L}\codeskip \Varid{x})\mathrel{=}\Conid{L}\codeskip (\Varid{imap}\codeskip \Varid{f}\codeskip \Varid{x}){}\<[E]%
\\
\>[3]{}\hsindent{2}{}\<[5]%
\>[5]{}\Varid{imap}\codeskip \Varid{f}\codeskip (\Conid{R}\codeskip \Varid{y})\mathrel{=}\Conid{R}\codeskip (\Varid{imap}\codeskip \Varid{f}\codeskip \Varid{y}){}\<[E]%
\ColumnHook
\end{hscode}\resethooks
\begin{hscode}\SaveRestoreHook
\column{B}{@{}>{\hspre}l<{\hspost}@{}}%
\column{3}{@{}>{\hspre}l<{\hspost}@{}}%
\column{19}{@{}>{\hspre}c<{\hspost}@{}}%
\column{19E}{@{}l@{}}%
\column{24}{@{}>{\hspre}l<{\hspost}@{}}%
\column{33}{@{}>{\hspre}c<{\hspost}@{}}%
\column{33E}{@{}l@{}}%
\column{38}{@{}>{\hspre}l<{\hspost}@{}}%
\column{46}{@{}>{\hspre}c<{\hspost}@{}}%
\column{46E}{@{}l@{}}%
\column{51}{@{}>{\hspre}l<{\hspost}@{}}%
\column{E}{@{}>{\hspre}l<{\hspost}@{}}%
\>[3]{}\mathbf{type}\codeskip \Conid{ParserF}_{10}{}\<[19]%
\>[19]{}\mathrel{=}{}\<[19E]%
\>[24]{}\Conid{Empty}_{10}{}\<[33]%
\>[33]{}:\!\!+\!\!:{}\<[33E]%
\>[38]{}\Conid{Pure}_{10}{}\<[46]%
\>[46]{}:\!\!+\!\!:{}\<[46E]%
\>[51]{}\Conid{Satisfy}_{10}{}\<[E]%
\\
\>[19]{}:\!\!+\!\!:{}\<[19E]%
\>[24]{}\Conid{Try}_{10}{}\<[33]%
\>[33]{}:\!\!+\!\!:{}\<[33E]%
\>[38]{}\Conid{Ap}_{10}{}\<[46]%
\>[46]{}:\!\!+\!\!:{}\<[46E]%
\>[51]{}\Conid{Or}_{10}{}\<[E]%
\\[\blanklineskip]%
\>[3]{}\mathbf{type}\codeskip \Conid{Parser}_{10}\mathrel{=}\Conid{Fix}\codeskip \Conid{ParserF}_{10}{}\<[E]%
\ColumnHook
\end{hscode}\resethooks
\begin{hscode}\SaveRestoreHook
\column{B}{@{}>{\hspre}l<{\hspost}@{}}%
\column{3}{@{}>{\hspre}l<{\hspost}@{}}%
\column{14}{@{}>{\hspre}l<{\hspost}@{}}%
\column{E}{@{}>{\hspre}l<{\hspost}@{}}%
\>[3]{}\Varid{aorb}_{10}\mathbin{::}\Conid{Parser}_{10}\codeskip \Conid{Char}{}\<[E]%
\\
\>[3]{}\Varid{aorb}_{10}\mathrel{=}\Conid{In}\codeskip (\Conid{R}\codeskip (\Conid{R}\codeskip (\Conid{R}\codeskip (\Conid{R}\codeskip (\Conid{R}\codeskip (\Conid{Or}_{10}{}\<[E]%
\\
\>[3]{}\hsindent{11}{}\<[14]%
\>[14]{}(\Conid{In}\codeskip (\Conid{R}\codeskip (\Conid{R}\codeskip (\Conid{L}\codeskip (\Conid{Satisfy}_{10}\codeskip (\equiv \text{\ttfamily{\textquotesingle}a\textquotesingle})))))){}\<[E]%
\\
\>[3]{}\hsindent{11}{}\<[14]%
\>[14]{}(\Conid{In}\codeskip (\Conid{R}\codeskip (\Conid{R}\codeskip (\Conid{L}\codeskip (\Conid{Satisfy}_{10}\codeskip (\equiv \text{\ttfamily{\textquotesingle}b\textquotesingle})))))))))))){}\<[E]%
\ColumnHook
\end{hscode}\resethooks


\begin{hscode}\SaveRestoreHook
\column{B}{@{}>{\hspre}l<{\hspost}@{}}%
\column{3}{@{}>{\hspre}l<{\hspost}@{}}%
\column{5}{@{}>{\hspre}l<{\hspost}@{}}%
\column{E}{@{}>{\hspre}l<{\hspost}@{}}%
\>[3]{}\mathbf{class}\codeskip (\Conid{IFunctor}\codeskip \Varid{f},\Conid{IFunctor}\codeskip \Varid{g})\Rightarrow \Varid{f}:\prec:\Varid{g}\codeskip \mathbf{where}{}\<[E]%
\\
\>[3]{}\hsindent{2}{}\<[5]%
\>[5]{}\Varid{inj}\mathbin{::}\Varid{f}\codeskip \Varid{k}\codeskip \Varid{a}\to \Varid{g}\codeskip \Varid{k}\codeskip \Varid{a}{}\<[E]%
\\[\blanklineskip]%
\>[3]{}\mathbf{instance}\codeskip \Conid{IFunctor}\codeskip \Varid{f}\Rightarrow \Varid{f}:\prec:\Varid{f}\codeskip \mathbf{where}{}\<[E]%
\\
\>[3]{}\hsindent{2}{}\<[5]%
\>[5]{}\Varid{inj}\mathrel{=}\Varid{id}{}\<[E]%
\\[\blanklineskip]%
\>[3]{}\mathbf{instance}\codeskip \mbox{\enskip\{-\# OVERLAPPING  \#-\}\enskip}(\Conid{IFunctor}\codeskip \Varid{f},\Conid{IFunctor}\codeskip \Varid{g})\Rightarrow \Varid{f}:\prec:(\Varid{f}:\!\!+\!\!:\Varid{g})\codeskip \mathbf{where}{}\<[E]%
\\
\>[3]{}\hsindent{2}{}\<[5]%
\>[5]{}\Varid{inj}\mathrel{=}\Conid{L}{}\<[E]%
\\[\blanklineskip]%
\>[3]{}\mathbf{instance}\codeskip (\Conid{IFunctor}\codeskip \Varid{f},\Conid{IFunctor}\codeskip \Varid{g},\Conid{IFunctor}\codeskip \Varid{h},\Varid{f}:\prec:\Varid{g})\Rightarrow \Varid{f}:\prec:(\Varid{h}:\!\!+\!\!:\Varid{g})\codeskip \mathbf{where}{}\<[E]%
\\
\>[3]{}\hsindent{2}{}\<[5]%
\>[5]{}\Varid{inj}\mathrel{=}\Conid{R}\hsdot{\cdot }{.}\Varid{inj}{}\<[E]%
\ColumnHook
\end{hscode}\resethooks
Smart constructors:
\begin{hscode}\SaveRestoreHook
\column{B}{@{}>{\hspre}l<{\hspost}@{}}%
\column{3}{@{}>{\hspre}l<{\hspost}@{}}%
\column{E}{@{}>{\hspre}l<{\hspost}@{}}%
\>[3]{}\Varid{empty}_{10}\mathbin{::}(\Conid{Empty}_{10}:\prec:\Varid{f})\Rightarrow \Conid{Fix}\codeskip \Varid{f}\codeskip \Varid{a}{}\<[E]%
\\
\>[3]{}\Varid{empty}_{10}\mathrel{=}\Conid{In}\codeskip (\Varid{inj}\codeskip \Conid{Empty}_{10}){}\<[E]%
\\[\blanklineskip]%
\>[3]{}\Varid{pure}_{10}\mathbin{::}(\Conid{Pure}_{10}:\prec:\Varid{f})\Rightarrow \Varid{a}\to \Conid{Fix}\codeskip \Varid{f}\codeskip \Varid{a}{}\<[E]%
\\
\>[3]{}\Varid{pure}_{10}\codeskip \Varid{x}\mathrel{=}\Conid{In}\codeskip (\Varid{inj}\codeskip (\Conid{Pure}_{10}\codeskip \Varid{x})){}\<[E]%
\\[\blanklineskip]%
\>[3]{}\Varid{satisfy}_{10}\mathbin{::}(\Conid{Satisfy}_{10}:\prec:\Varid{f})\Rightarrow (\Conid{Char}\to \Conid{Bool})\to \Conid{Fix}\codeskip \Varid{f}\codeskip \Conid{Char}{}\<[E]%
\\
\>[3]{}\Varid{satisfy}_{10}\codeskip \Varid{c}\mathrel{=}\Conid{In}\codeskip (\Varid{inj}\codeskip (\Conid{Satisfy}_{10}\codeskip \Varid{c})){}\<[E]%
\\[\blanklineskip]%
\>[3]{}\Varid{try}_{10}\mathbin{::}(\Conid{Try}_{10}:\prec:\Varid{f})\Rightarrow \Conid{Fix}\codeskip \Varid{f}\codeskip \Varid{a}\to \Conid{Fix}\codeskip \Varid{f}\codeskip \Varid{a}{}\<[E]%
\\
\>[3]{}\Varid{try}_{10}\codeskip \Varid{px}\mathrel{=}\Conid{In}\codeskip (\Varid{inj}\codeskip (\Conid{Try}_{10}\codeskip \Varid{px})){}\<[E]%
\\[\blanklineskip]%
\>[3]{}\Varid{ap}_{10}\mathbin{::}(\Conid{Ap}_{10}:\prec:\Varid{f})\Rightarrow \Conid{Fix}\codeskip \Varid{f}\codeskip (\Varid{a}\to \Varid{b})\to \Conid{Fix}\codeskip \Varid{f}\codeskip \Varid{a}\to \Conid{Fix}\codeskip \Varid{f}\codeskip \Varid{b}{}\<[E]%
\\
\>[3]{}\Varid{ap}_{10}\codeskip \Varid{pf}\codeskip \Varid{px}\mathrel{=}\Conid{In}\codeskip (\Varid{inj}\codeskip (\Conid{Ap}_{10}\codeskip \Varid{pf}\codeskip \Varid{px})){}\<[E]%
\\[\blanklineskip]%
\>[3]{}\Varid{or}_{10}\mathbin{::}(\Conid{Or}_{10}:\prec:\Varid{f})\Rightarrow \Conid{Fix}\codeskip \Varid{f}\codeskip \Varid{a}\to \Conid{Fix}\codeskip \Varid{f}\codeskip \Varid{a}\to \Conid{Fix}\codeskip \Varid{f}\codeskip \Varid{a}{}\<[E]%
\\
\>[3]{}\Varid{or}_{10}\codeskip \Varid{px}\codeskip \Varid{py}\mathrel{=}\Conid{In}\codeskip (\Varid{inj}\codeskip (\Conid{Or}_{10}\codeskip \Varid{px}\codeskip \Varid{py})){}\<[E]%
\ColumnHook
\end{hscode}\resethooks

\begin{hscode}\SaveRestoreHook
\column{B}{@{}>{\hspre}l<{\hspost}@{}}%
\column{3}{@{}>{\hspre}l<{\hspost}@{}}%
\column{E}{@{}>{\hspre}l<{\hspost}@{}}%
\>[3]{}\Varid{aorb}_{10}'\mathbin{::}(\Conid{Or}_{10}:\prec:\Varid{f},\Conid{Satisfy}_{10}:\prec:\Varid{f})\Rightarrow \Conid{Fix}\codeskip \Varid{f}\codeskip \Conid{Char}{}\<[E]%
\\
\>[3]{}\Varid{aorb}_{10}'\mathrel{=}\Varid{satisfy}_{10}\codeskip (\equiv \text{\ttfamily{\textquotesingle}a\textquotesingle})\mathbin{\text{\`{}}\Varid{or}_{10}\text{\`{}}}\Varid{satisfy}_{10}\codeskip (\equiv \text{\ttfamily{\textquotesingle}b\textquotesingle}){}\<[E]%
\ColumnHook
\end{hscode}\resethooks


\begin{hscode}\SaveRestoreHook
\column{B}{@{}>{\hspre}l<{\hspost}@{}}%
\column{3}{@{}>{\hspre}l<{\hspost}@{}}%
\column{5}{@{}>{\hspre}l<{\hspost}@{}}%
\column{E}{@{}>{\hspre}l<{\hspost}@{}}%
\>[3]{}\mathbf{class}\codeskip \Conid{IFunctor}\codeskip \Varid{f}\Rightarrow \Conid{SizeAlg}\codeskip \Varid{f}\codeskip \mathbf{where}{}\<[E]%
\\
\>[3]{}\hsindent{2}{}\<[5]%
\>[5]{}\Varid{sizeAlg}_{10}\mathbin{::}\Varid{f}\codeskip \Conid{Size}_{8}\codeskip \Varid{i}\to \Conid{Size}_{8}\codeskip \Varid{i}{}\<[E]%
\\[\blanklineskip]%
\>[3]{}\mathbf{instance}\codeskip (\Conid{SizeAlg}\codeskip \Varid{f},\Conid{SizeAlg}\codeskip \Varid{g})\Rightarrow \Conid{SizeAlg}\codeskip (\Varid{f}:\!\!+\!\!:\Varid{g})\codeskip \mathbf{where}{}\<[E]%
\\
\>[3]{}\hsindent{2}{}\<[5]%
\>[5]{}\Varid{sizeAlg}_{10}\codeskip (\Conid{L}\codeskip \Varid{x})\mathrel{=}\Varid{sizeAlg}_{10}\codeskip \Varid{x}{}\<[E]%
\\
\>[3]{}\hsindent{2}{}\<[5]%
\>[5]{}\Varid{sizeAlg}_{10}\codeskip (\Conid{R}\codeskip \Varid{y})\mathrel{=}\Varid{sizeAlg}_{10}\codeskip \Varid{y}{}\<[E]%
\\[\blanklineskip]%
\>[3]{}\mathbf{instance}\codeskip \Conid{SizeAlg}\codeskip \Conid{Or}_{10}\codeskip \mathbf{where}{}\<[E]%
\\
\>[3]{}\hsindent{2}{}\<[5]%
\>[5]{}\Varid{sizeAlg}_{10}\codeskip (\Conid{Or}_{10}\codeskip \Varid{px}\codeskip \Varid{py})\mathrel{=}\Varid{px}\mathbin{+}\Varid{py}\mathbin{+}\mathrm{1}{}\<[E]%
\\[\blanklineskip]%
\>[3]{}\mathbf{instance}\codeskip \Conid{SizeAlg}\codeskip \Conid{Satisfy}_{10}\codeskip \mathbf{where}{}\<[E]%
\\
\>[3]{}\hsindent{2}{}\<[5]%
\>[5]{}\Varid{sizeAlg}_{10}\codeskip (\Conid{Satisfy}_{10}\codeskip \anonymous )\mathrel{=}\mathrm{1}{}\<[E]%
\ColumnHook
\end{hscode}\resethooks
\begin{hscode}\SaveRestoreHook
\column{B}{@{}>{\hspre}l<{\hspost}@{}}%
\column{3}{@{}>{\hspre}l<{\hspost}@{}}%
\column{E}{@{}>{\hspre}l<{\hspost}@{}}%
\>[3]{}\Varid{size}_{10}\mathbin{::}\Conid{SizeAlg}\codeskip \Varid{f}\Rightarrow \Conid{Fix}\codeskip \Varid{f}\codeskip \Varid{a}\to \Conid{Size}_{8}\codeskip \Varid{a}{}\<[E]%
\\
\>[3]{}\Varid{size}_{10}\mathrel{=}\Varid{cata}\codeskip \Varid{sizeAlg}_{10}{}\<[E]%
\\[\blanklineskip]%
\>[3]{}\Varid{eval}\mathbin{::}\Conid{Size}{}\<[E]%
\\
\>[3]{}\Varid{eval}\mathrel{=}\Varid{coerce}\codeskip (\Varid{size}_{10}\codeskip (\Varid{aorb}_{10}'\mathbin{::}(\Conid{Fix}\codeskip (\Conid{Or}_{10}:\!\!+\!\!:\Conid{Satisfy}_{10}))\codeskip \Conid{Char})){}\<[E]%
\ColumnHook
\end{hscode}\resethooks




\bibliography{biblo}

\appendix
\section{Appendix}

\begin{hscode}\SaveRestoreHook
\column{B}{@{}>{\hspre}l<{\hspost}@{}}%
\column{3}{@{}>{\hspre}l<{\hspost}@{}}%
\column{9}{@{}>{\hspre}l<{\hspost}@{}}%
\column{23}{@{}>{\hspre}c<{\hspost}@{}}%
\column{23E}{@{}l@{}}%
\column{26}{@{}>{\hspre}l<{\hspost}@{}}%
\column{31}{@{}>{\hspre}l<{\hspost}@{}}%
\column{40}{@{}>{\hspre}l<{\hspost}@{}}%
\column{E}{@{}>{\hspre}l<{\hspost}@{}}%
\>[3]{}\mathbf{type}\codeskip \Conid{Size}\mathrel{=}\Conid{Int}{}\<[E]%
\\
\>[3]{}\Varid{size}\mathbin{::}\Conid{Parser}_{2}\codeskip \Varid{a}\to \Conid{Size}{}\<[E]%
\\
\>[3]{}\Varid{size}\codeskip {}\<[9]%
\>[9]{}\Conid{Empty}_{2}{}\<[23]%
\>[23]{}\mathrel{=}{}\<[23E]%
\>[26]{}\mathrm{1}{}\<[E]%
\\
\>[3]{}\Varid{size}\codeskip {}\<[9]%
\>[9]{}(\Conid{Pure}_{2}\codeskip \anonymous ){}\<[23]%
\>[23]{}\mathrel{=}{}\<[23E]%
\>[26]{}\mathrm{1}{}\<[E]%
\\
\>[3]{}\Varid{size}\codeskip {}\<[9]%
\>[9]{}(\Conid{Satisfy}_{2}\codeskip \anonymous ){}\<[23]%
\>[23]{}\mathrel{=}{}\<[23E]%
\>[26]{}\mathrm{1}{}\<[E]%
\\
\>[3]{}\Varid{size}\codeskip {}\<[9]%
\>[9]{}(\Conid{Try}_{2}\codeskip \Varid{px}){}\<[23]%
\>[23]{}\mathrel{=}{}\<[23E]%
\>[26]{}\mathrm{1}\mathbin{+}{}\<[31]%
\>[31]{}\Varid{size}\codeskip \Varid{px}{}\<[E]%
\\
\>[3]{}\Varid{size}\codeskip {}\<[9]%
\>[9]{}(\Conid{Ap}_{2}\codeskip \Varid{pf}\codeskip \Varid{px}){}\<[23]%
\>[23]{}\mathrel{=}{}\<[23E]%
\>[26]{}\mathrm{1}\mathbin{+}{}\<[31]%
\>[31]{}\Varid{size}\codeskip \Varid{pf}{}\<[40]%
\>[40]{}\mathbin{+}\Varid{size}\codeskip \Varid{px}{}\<[E]%
\\
\>[3]{}\Varid{size}\codeskip {}\<[9]%
\>[9]{}(\Conid{Or}_{2}\codeskip \Varid{px}\codeskip \Varid{py}){}\<[23]%
\>[23]{}\mathrel{=}{}\<[23E]%
\>[26]{}\mathrm{1}\mathbin{+}{}\<[31]%
\>[31]{}\Varid{size}\codeskip \Varid{px}{}\<[40]%
\>[40]{}\mathbin{+}\Varid{size}\codeskip \Varid{py}{}\<[E]%
\ColumnHook
\end{hscode}\resethooks
\begin{hscode}\SaveRestoreHook
\column{B}{@{}>{\hspre}l<{\hspost}@{}}%
\column{3}{@{}>{\hspre}l<{\hspost}@{}}%
\column{E}{@{}>{\hspre}l<{\hspost}@{}}%
\>[3]{}\mathbf{type}\codeskip \Conid{Parser}_{3}\codeskip \Varid{a}\mathrel{=}\Conid{Int}{}\<[E]%
\\
\>[3]{}\Varid{pure}_{3}\codeskip \anonymous \mathrel{=}\mathrm{1}{}\<[E]%
\\
\>[3]{}\Varid{satisfy}_{3}\codeskip \anonymous \mathrel{=}\mathrm{1}{}\<[E]%
\\
\>[3]{}\Varid{empty}_{3}\mathrel{=}\mathrm{1}{}\<[E]%
\\
\>[3]{}\Varid{try}_{3}\codeskip \Varid{px}\mathrel{=}\Varid{px}\mathbin{+}\mathrm{1}{}\<[E]%
\\
\>[3]{}\Varid{ap}_{3}\codeskip \Varid{pf}\codeskip \Varid{px}\mathrel{=}\Varid{pf}\mathbin{+}\Varid{pf}\mathbin{+}\mathrm{1}{}\<[E]%
\\
\>[3]{}\Varid{or}_{3}\codeskip \Varid{px}\codeskip \Varid{py}\mathrel{=}\Varid{px}\mathbin{+}\Varid{py}\mathbin{+}\mathrm{1}{}\<[E]%
\\[\blanklineskip]%
\>[3]{}\Varid{size}_{3}\mathbin{::}\Conid{Parser}_{3}\codeskip \Varid{a}\to \Conid{Size}{}\<[E]%
\\
\>[3]{}\Varid{size}_{3}\mathrel{=}\Varid{id}{}\<[E]%
\ColumnHook
\end{hscode}\resethooks
\begin{hscode}\SaveRestoreHook
\column{B}{@{}>{\hspre}l<{\hspost}@{}}%
\column{3}{@{}>{\hspre}l<{\hspost}@{}}%
\column{5}{@{}>{\hspre}l<{\hspost}@{}}%
\column{E}{@{}>{\hspre}l<{\hspost}@{}}%
\>[3]{}\mathbf{class}\codeskip \Conid{IFunctor}\codeskip \Varid{f}\codeskip \mathbf{where}{}\<[E]%
\\
\>[3]{}\hsindent{2}{}\<[5]%
\>[5]{}\Varid{imap}\mathbin{::}(\forall \Varid{i}\hsforall \hsdot{\cdot }{.}\Varid{a}\codeskip \Varid{i}\to \Varid{b}\codeskip \Varid{i})\to \Varid{f}\codeskip \Varid{a}\codeskip \Varid{i}\to \Varid{f}\codeskip \Varid{b}\codeskip \Varid{i}{}\<[E]%
\\[\blanklineskip]%
\>[3]{}\mathbf{newtype}\codeskip \Conid{Fix}\codeskip \Varid{f}\codeskip \Varid{a}\mathrel{=}\Conid{In}\codeskip (\Varid{f}\codeskip (\Conid{Fix}\codeskip \Varid{f})\codeskip \Varid{a}){}\<[E]%
\\
\>[3]{}\Varid{cata}\mathbin{::}\Conid{IFunctor}\codeskip \Varid{f}\Rightarrow (\forall \Varid{i}\hsforall \hsdot{\cdot }{.}\Varid{f}\codeskip \Varid{a}\codeskip \Varid{i}\to \Varid{a}\codeskip \Varid{i})\to \Conid{Fix}\codeskip \Varid{f}\codeskip \Varid{i}\to \Varid{a}\codeskip \Varid{i}{}\<[E]%
\\
\>[3]{}\Varid{cata}\codeskip \Varid{alg}\codeskip (\Conid{In}\codeskip \Varid{x})\mathrel{=}\Varid{alg}\codeskip (\Varid{imap}\codeskip (\Varid{cata}\codeskip \Varid{alg})\codeskip \Varid{x}){}\<[E]%
\ColumnHook
\end{hscode}\resethooks
\begin{hscode}\SaveRestoreHook
\column{B}{@{}>{\hspre}l<{\hspost}@{}}%
\column{3}{@{}>{\hspre}l<{\hspost}@{}}%
\column{5}{@{}>{\hspre}l<{\hspost}@{}}%
\column{E}{@{}>{\hspre}l<{\hspost}@{}}%
\>[3]{}\mathbf{instance}\codeskip \Conid{IFunctor}\codeskip \Conid{Empty}_{10}\codeskip \mathbf{where}{}\<[E]%
\\
\>[3]{}\hsindent{2}{}\<[5]%
\>[5]{}\Varid{imap}\codeskip \anonymous \codeskip \Conid{Empty}_{10}\mathrel{=}\Conid{Empty}_{10}{}\<[E]%
\\[\blanklineskip]%
\>[3]{}\mathbf{instance}\codeskip \Conid{IFunctor}\codeskip \Conid{Pure}_{10}\codeskip \mathbf{where}{}\<[E]%
\\
\>[3]{}\hsindent{2}{}\<[5]%
\>[5]{}\Varid{imap}\codeskip \anonymous \codeskip (\Conid{Pure}_{10}\codeskip \Varid{x})\mathrel{=}\Conid{Pure}_{10}\codeskip \Varid{x}{}\<[E]%
\\[\blanklineskip]%
\>[3]{}\mathbf{instance}\codeskip \Conid{IFunctor}\codeskip \Conid{Satisfy}_{10}\codeskip \mathbf{where}{}\<[E]%
\\
\>[3]{}\hsindent{2}{}\<[5]%
\>[5]{}\Varid{imap}\codeskip \anonymous \codeskip (\Conid{Satisfy}_{10}\codeskip \Varid{c})\mathrel{=}\Conid{Satisfy}_{10}\codeskip \Varid{c}{}\<[E]%
\\[\blanklineskip]%
\>[3]{}\mathbf{instance}\codeskip \Conid{IFunctor}\codeskip \Conid{Try}_{10}\codeskip \mathbf{where}{}\<[E]%
\\
\>[3]{}\hsindent{2}{}\<[5]%
\>[5]{}\Varid{imap}\codeskip \Varid{f}\codeskip (\Conid{Try}_{10}\codeskip \Varid{px})\mathrel{=}\Conid{Try}_{10}\mathbin{\$}\Varid{f}\codeskip \Varid{px}{}\<[E]%
\\[\blanklineskip]%
\>[3]{}\mathbf{instance}\codeskip \Conid{IFunctor}\codeskip \Conid{Ap}_{10}\codeskip \mathbf{where}{}\<[E]%
\\
\>[3]{}\hsindent{2}{}\<[5]%
\>[5]{}\Varid{imap}\codeskip \Varid{f}\codeskip (\Conid{Ap}_{10}\codeskip \Varid{pf}\codeskip \Varid{px})\mathrel{=}\Conid{Ap}_{10}\codeskip (\Varid{f}\codeskip \Varid{pf})\codeskip (\Varid{f}\codeskip \Varid{px}){}\<[E]%
\\[\blanklineskip]%
\>[3]{}\mathbf{instance}\codeskip \Conid{IFunctor}\codeskip \Conid{Or}_{10}\codeskip \mathbf{where}{}\<[E]%
\\
\>[3]{}\hsindent{2}{}\<[5]%
\>[5]{}\Varid{imap}\codeskip \Varid{f}\codeskip (\Conid{Or}_{10}\codeskip \Varid{px}\codeskip \Varid{py})\mathrel{=}\Conid{Or}_{10}\codeskip (\Varid{f}\codeskip \Varid{px})\codeskip (\Varid{f}\codeskip \Varid{py}){}\<[E]%
\ColumnHook
\end{hscode}\resethooks

\end{document}
